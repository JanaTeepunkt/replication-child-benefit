\documentclass[11pt, a4paper, leqno]{article}
\usepackage{a4wide}
\usepackage[T1]{fontenc}
\usepackage[utf8]{inputenc}
\usepackage{float, afterpage, rotating, graphicx}
\usepackage{epstopdf}
\usepackage{longtable, booktabs, tabularx}
\usepackage{fancyvrb, moreverb, relsize}
\usepackage{eurosym, calc}
% \usepackage{chngcntr}
\usepackage{amsmath, amssymb, amsfonts, amsthm, bm}
\usepackage{caption}
\usepackage{mdwlist}
\usepackage{xfrac}
\usepackage{setspace}
\usepackage[dvipsnames]{xcolor}
\usepackage{subcaption}
\usepackage{minibox}
% \usepackage{pdf14} % Enable for Manuscriptcentral -- can't handle pdf 1.5
% \usepackage{endfloat} % Enable to move tables / figures to the end. Useful for some
% submissions.

\usepackage[
    natbib=true,
    bibencoding=inputenc,
    bibstyle=authoryear-ibid,
    citestyle=authoryear-comp,
    maxcitenames=3,
    maxbibnames=10,
    useprefix=false,
    sortcites=true,
    backend=biber
]{biblatex}
\AtBeginDocument{\toggletrue{blx@useprefix}}
\AtBeginBibliography{\togglefalse{blx@useprefix}}
\setlength{\bibitemsep}{1.5ex}
\addbibresource{../../paper/refs.bib}

\usepackage[unicode=true]{hyperref}
\hypersetup{
    colorlinks=true,
    linkcolor=black,
    anchorcolor=black,
    citecolor=NavyBlue,
    filecolor=black,
    menucolor=black,
    runcolor=black,
    urlcolor=NavyBlue
}


\widowpenalty=10000
\clubpenalty=10000

\setlength{\parskip}{1ex}
\setlength{\parindent}{0ex}
\setstretch{1.5}


\begin{document}

\title{Replication of "The Effect of a Universal Child Benefit on Conceptions, Abortions, and Early Maternal Labor Supply"
by Libertad Gonzalez (2013)\thanks{Jana Tissen, University of Cologne. Email: \href{mailto:jana.tissen@gmail.com}{\nolinkurl{jana [dot] tissen [at] gmail [dot] com}}.}}

\author{Jana Tissen}

\date{
    {\bf Preliminary -- please do not quote}
    \\[1ex]
    \today
}

\maketitle


\begin{abstract}
This project replicates the main regressions from the paper \citet{gonzalez2013effect}.
\end{abstract}

\clearpage


\section{Summary} % (fold)
\label{sec:summary}

This project replicates the main graph and the main regressions of the paper "The Effect of a Universal Child Benefit on Conceptions, Abortions, and Early Maternal Labor Supply"
by Libertad Gonzalez (2013)\footnote{This project was built with help of the following template: \citet{GaudeckerEconProjectTemplates}.}.
In July 2007 the Spanish government implemented a one-time child benefit payment for children that
were born after July 1st 2007. The goal was to increase fertility by compensating the increase in expenditures.
This payment was not announced beforehand. \citet{gonzalez2013effect} uses a regression discontinuity design (rdd) to estimate the causal influence
of this benefit on conceptions and abortions.
In this project I focus on replicating the influence on conceptions.

The data set for this project is taken from
\url{https://www.openicpsr.org/openicpsr/project/114827/version/V1/view?path=/pcms/projects/1/1/4/8/114827/V1.0.1/Data_20110196/data_births_20110196.dta}\footnote{The stata dofile can be downloaded here:
\url{https://www.openicpsr.org/openicpsr/project/114827/version/V1/view?path=/pcms/projects/1/1/4/8/114827/V1.0.1/Data_20110196/dofile_fertility_20110196.do}.}.
It contains 4984066 observations of mothers in Spain who gave birth between 2000 and 2011.
For the main analysis four original variables are used, the others are created in the data management process.
\texttt{mesp} and \texttt{year} are the month and the year of giving birth, \texttt{prem} is a dummy whether it was a premature
birth (=1 if false, =2 if true) and \texttt{semanas} in which pregnancy week the birth was.
During the data management process a running variable \texttt{month} for the month of birth and a running variable \texttt{m}
for month of conception is estimated by subtracting \texttt{semanas} from \texttt{month}.

The figure illustrates how conception rates changed before and after the cutoff date.
Five regressions are estimated within different time periods. The first between 2000 and 2009, the second between 2005 and 2009,
and the others 12, 9 and 3 months around the cutoff date.
The regression equation is

\begin{equation} \label{eq:main}
    \boldsymbol{C}_m= \alpha + \beta post + \sum_{p=1}^{P} (\gamma_p m^p + \delta_p m^p post)  + \lambda \boldsymbol{X}_m + \epsilon_m
\end{equation}

where $\boldsymbol{C}_m$ is the log-number of conceptions in month $m$, month $m$ is the month of conception which is
normalized to be 0 in July 2007. $post$ is the treatment variable, a dummy which equals 1 for every month after June 2007.
$\boldsymbol{X}_m$ serves as control variable for different month lengths and is the number of days in month $m$. $\epsilon_m$
is a random error term. $\beta$ is the coefficient of interest that indicates a jump in the number of conceptions at the threshold.


\begin{figure}[H]

    \centering
    \includegraphics[width=0.85\textwidth]{../bld/figures/bimonthly_conc}

    \caption{Bimonthly conception rates in Spain between 2005 and 2009}
    \label{fig:conc}
    \begin{footnotesize} \textit{Notes:} The graph shows the bimonthly average of conceptions in Spain and separated linear fits for before
        and after July 2007. July/August dots are shown in yellow.
    \end{footnotesize}

\end{figure}


\begin{table}[!h]
    \input{../bld/tables/rdd_results_0.tex}
    \caption{\label{tab:reg0} Estimation results for rdd-model for conceptions between 2000 and 2009.}
\end{table}

\begin{table}[!h]
    \input{../bld/tables/rdd_results_1.tex}
    \caption{\label{tab:reg1} Estimation results for rdd-model for conceptions between 2005 and 2009.}
\end{table}

\begin{table}[!h]
    \input{../bld/tables/rdd_results_2.tex}
    \caption{\label{tab:reg2} Estimation results for rdd-model for conceptions 12 months before and after July 2007.}
\end{table}

\begin{table}[!h]
    \input{../bld/tables/rdd_results_3.tex}
    \caption{\label{tab:reg3} Estimation results for rdd-model for conceptions 9 months before and after July 2007.}
\end{table}

\begin{table}[!h]
    \input{../bld/tables/rdd_results_4.tex}
    \caption{\label{tab:reg4} Estimation results for rdd-model for conceptions 3 months before and after July 2007.}
\end{table}

% section introduction (end)


\clearpage
\setstretch{1}
\printbibliography
\setstretch{1.5}


% \appendix

% The chngctr package is needed for the following lines.
% \counterwithin{table}{section}
% \counterwithin{figure}{section}

\end{document}
